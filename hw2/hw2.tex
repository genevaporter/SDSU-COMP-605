\documentclass[12pt]{article}
\usepackage[utf8]{inputenc}
\usepackage{amsmath}
\usepackage{nopageno}
\usepackage[margin=1.15in]{geometry}

\pagestyle{plain}

\title{Computational Science 605: Homework 2}
\author{Geneva Porter}
\date{26 March 2020}

\begin{document}
%\maketitle
\begin{center}
	{\Huge COMP-605 Homework 2}
	
	\vspace{5mm}
	
	{\Large Dr. J Corbino, SDSU}
	
	\vspace{5mm}
	
	{\large Geneva Porter, February 24 2020}
	
	\vspace{5mm}
\end{center}

For this assignment, I used the cluster to run parallel threads using Open MP. Below is an example table of one of the iterations:

\begin{table}[h]
    \centering
    \begin{tabular}{|c|c|c|} \hline
        ~ & N & Approximation & Thread # used when printing \\ \hline
         No optimization (-O0 flag)&  0.0023799 & 0.00275436\\ \hline
         With optimization (-O3 flag)& 0.00051495 & 0.000476634 \\ \hline
    \end{tabular}
	\caption{Run times for $50\times50$ matrix}
\end{table}

We can see that as N increases, the approximation value tends toward zero. I tried this code using 1 billion iterations, and got an approximation of $\pi$ to 6 decimal places (I expected musch higher accuracy for that number of iterations, but oh well). Also, since I used different loops for parallelization, I only captured the thread number used for printing the table of results. The desired number of threads can be passed as an argument when running the .cpp file.

Memory leaks were checked, and although valgrind threw some errors regarding initialization, all heap blocks were freed and no leaks were possible. I struggled with choosing which omp modifiers to use that would best parallelize my code, however I found that I had to break it up into smaller loops to work.

\end{document}