\documentclass{article}
\usepackage[utf8]{inputenc}

\title{Computational Science 605: \\ Scientific Computing and Parallel Processing}
\author{Geneva Porter}
\date{Spring 2020}

\begin{document}

\maketitle

\section{Accessing the Cluster}

As of 2/6, we can only access the computing cluster while physically on the SDSU campus. From a terminal on campus, type:

\begin{verbatim}
$ ssh porter@tuckoo.sdsu.edu
\end{verbatim}

\noindent My default password is 

\begin{verbatim}
Porter_2020_SDSU
\end{verbatim}

\noindent Use nano editor (has shortcut commands to follow). We can create a new document with nano as well. Here's a quick axample of how to make, compile, and run a c++ program:

\begin{verbatim}
$ nano hello.cpp
\end{verbatim}

\noindent In the editor that appears, write the program:

\begin{verbatim}
#include <iostream>

int main{
    std::cout << "hello" << '\n';
    return 0;
}
\end{verbatim}

\noindent Press ctrl + O to save, then ctrl + X to exit. To compile and run, use the command prompt to write:

\begin{verbatim}
c++ hello.cpp
./a.out
\end{verbatim}

\end{document}














