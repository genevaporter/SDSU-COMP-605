\documentclass[12pt]{article}
\usepackage[utf8]{inputenc}
\usepackage{amsmath}
\usepackage{nopageno}
\usepackage[margin=1.15in]{geometry}

\pagestyle{plain}

\title{Computational Science 605: Homework 0}
\author{Geneva Porter}
\date{24 February 2020}

\begin{document}
%\maketitle
\begin{center}
	{\Huge COMP-605 Homework 0}
	
	\vspace{5mm}
	
	{\Large Dr. J Corbino, SDSU}
	
	\vspace{5mm}
	
	{\large Geneva Porter, February 24 2020}
	
	\vspace{5mm}
\end{center}

For this assignment, a native Ubuntu 64-bit environment was used on a machine with 16 GB memory and 2.7 GHz CPU. I used C++ as my coding language. The solutions are written to output-solutions.txt upon compiling and running the code, Porter-hw00.cpp.

For measuring run times for each scheme, I ran each algorithm 1000 times and summed the resulting time measurements. From this I was able to take the average and  provide a reasonably accurate estimate of the optimization effects. One odd occurrence is that the matrix multiplication time would occasionally output a negative value; in this case I ignored that value and ran the loop an additional time. This may be due to a clock resolution issue on my machine. 


\begin{table}[h]
    \centering
    \begin{tabular}{|c|c|c|} \hline
        ~ & ijk-form & jki-form \\ \hline
         No optimization (-O0 flag)&  0.0023799 & 0.00275436\\ \hline
         With optimization (-O3 flag)& 0.00051495 & 0.000476634 \\ \hline
    \end{tabular}
	\caption{Run times for $50\times50$ matrix}
\end{table}

\begin{table}[h]
	\centering
	\begin{tabular}{|c|c|c|} \hline
		~ & ijk-form & jki-form \\ \hline
		No optimization (-O0 flag)&  0.0209508 & 0.0225209\\ \hline
		With optimization (-O3 flag)& 0.00483495 & 0.0042901 \\ \hline
	\end{tabular}
	\caption{Run times for $100\times100$ matrix}
\end{table}

\begin{table}[h]
	\centering
	\begin{tabular}{|c|c|c|} \hline
		~ & ijk-form & jki-form \\ \hline
		No optimization (-O0 flag)&  0.12632 & 0.139251 \\ \hline
		With optimization (-O3 flag)& 0.0285915 & 0.0363679 \\ \hline
	\end{tabular}
	\caption{Run times for $200\times300$ matrix}
\end{table}

Since I used C++, the ijk row algorithm is predictably more efficient for the non-optimized run. This is because C++ accesses arrays through storing values in rows, which is more intuitive (for myself, as an applied math student). However, after testing several times, the optimized run shows that the jki column algorithm is more time efficient for the $50\times50$ and $100\times100$ matrices, but less efficient for the $200\times300$ matrix. I'm not sure why this switch happened. 

Memory leaks were checked, and although valgrind threw some errors regarding initialization, all heap blocks were freed and no leaks were possible. While the instructions stipulated that 40,000 32-bit real values should be stored, since I allocated a 2D array, I limited the matrix dimensions to 2000, which allowed for 4,000,000 entries. Hopefully this is sufficient for the assignment requirements and not wasteful.

\end{document}